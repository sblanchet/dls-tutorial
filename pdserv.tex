\chapter{PdServ}


PdServ is the library that implements the MSR protocol on the server side.

\section{Install prerequesites}

The following prerequesites must be installed to compile PdServ and its test
server.

\noindent Open a terminal and type (the ending backslash means that
the command continues on the next line).
\begin{verbatim}
sudo apt install mercurial cmake g++ pkg-config     \
                 libdb-dev libsasl2-dev libyaml-dev \
                 liblog4cplus-dev libcommoncpp2-dev
\end{verbatim}


\section{Get PdServ source code}
Download the source code with mercurial\footnote{The Mercurial main
  command tool is named \texttt{hg} because the chemical symbol for
  mercury is \emph{Hg}. Mercurial is a source control management tool
  very similar to \texttt{git}.}, then switch to the revision that has
been tested in this tutorial (rev 528).

\begin{verbatim}
cd /tmp
hg clone http://hg.code.sf.net/p/pdserv/code pdserv-code
cd /tmp/pdserv-code
hg update --rev 528
\end{verbatim}

\section{Compile PdServ source code}

PdServ needs \texttt{cmake} to create the Makefile.\\
Do not forget the dot just after \texttt{cmake}
\begin{verbatim}
cd /tmp/pdserv-code
cmake . -DCMAKE_INSTALL_PREFIX=/opt/etherlab
make
\end{verbatim}


\section{Install PdServ library}
The library will be installed in \texttt{/opt/etherlab} \\
Become root with \texttt{sudo -i} then run the installation commands

\begin{verbatim}
sudo -i
cd /tmp/pdserv-code
make install
echo /opt/etherlab/lib > /etc/ld.so.conf.d/etherlab.conf
/sbin/ldconfig
ln -sf /opt/etherlab/lib/pkgconfig/libpdserv.pc /usr/share/pkgconfig/
\end{verbatim}


\section{Compile a test server}
\label{test-server-for-testmanager}
PdServ comes with a test server, but it needs to be patched
to have more visible signals in the viewer.

\noindent Edit the file \texttt{/tmp/pdserv-code/test/test1.cpp} with
\texttt{gedit},
\begin{verbatim}
gedit /tmp/pdserv-code/test/test1.cpp
\end{verbatim}

\noindent Go to line 165 to add at the beginning of the while loop.

\begin{verbatim}
        s1.c = (s1.c + 1) % 20;
\end{verbatim}

\noindent Check the modification with mercurial


\begin{verbatim}
cd /tmp/pdserv-code
hg diff
diff -r 90fefe8b0e99 test/test1.cpp
--- a/test/test1.cpp    Sat Nov 26 15:45:47 2016 +0100
+++ b/test/test1.cpp    Mon Dec 17 13:08:36 2018 +0100
@@ -162,6 +162,7 @@

     int i;
     while (1) {
+       s1.c = (s1.c + 1) % 20;
         //sleep(1);
         usleep(100000);
         clock_gettime(CLOCK_REALTIME, &time);
\end{verbatim}

\noindent Compile the test server

\begin{verbatim}
g++ -o test1 test1.cpp -I/opt/etherlab/include -L/opt/etherlab/lib -lpdserv
\end{verbatim}

\noindent Run the program
\begin{verbatim}
./test1
\end{verbatim}

\noindent An error message appears but it is not important for this tutorial.
\begin{verbatim}
Error loading default config file /etc/opt/etherlab/pdserv.conf :
: File is not readable
\end{verbatim}

\noindent Keep this program running until the tutorial end
to have a test server for TestManager and DLS.
