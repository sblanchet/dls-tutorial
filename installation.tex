\chapter{Installation}


This chapter explains how to install DLS on a virtual machine running Ubuntu 18.04 LTS amd64.


\subsubsection{Notes}

\begin{itemize}
\item For a very first installation, it is recommended to use exactly the same
  distribution than this guide to avoid any distribution issues.
\item For simplicity, all the components will be installed on the same host,
  but obviously the server and client can run on different machines.
\end{itemize}



\section{Creation of the virtual machine}


Create a virtual machine with your favorite virtualisation tool (for example VirtualBox)


Recommended settings:
\begin{itemize}
\item vCPU: 1
\item RAM: 4 GB
\item vDisk: 20 GB
\item Guest OS: Ubuntu 18.04 64 bits
\end{itemize}


Download \texttt{ubuntu-18.04-desktop-amd64.iso} from
\url{https://www.ubuntu.com}
and install it in your new virtual machine.

Select a minimum installation with only a web browser and basic utilities
to save disk space.



\subsection{Install prerequesites}


You must install the following prerequesites to compiles dls and its dependencies

\begin{verbatim}
sudo apt install mercurial cmake g++ pkg-config      \
                 libdb-dev libsasl2-dev  libyaml-dev \
                 liblog4cplus-dev libcommoncpp2-dev
\end{verbatim}



\subsection{Install pdserv}

PdServ is the library that implements the MSR protocol on the server side.
PdServ is not mandatory for DLS, nethertheless it is neeeded in the case of this tutorial to create a small server application to test DLS.


\subsubsection{Get PdServ source code}
Download the source code with mercurial, then switch to the revision
that has been tested in this tutorial (528).
\begin{verbatim}
hg clone http://hg.code.sf.net/p/pdserv/code pdserv-code
cd pdserv-code
hg update -r528
\end{verbatim}

\subsubsection{Compile PdServ source code}

Do not forget the dot just after \texttt{cmake}
\begin{verbatim}
cmake . -DCMAKE_INSTALL_PREFIX:PATH=/opt/etherlab -DCMAKE_INSTALL_LIBDIR=lib
make
\end{verbatim}


\subsubsection{Install PdServ library}
The library is installed in \texttt{/opt/etherlab} \\
Become root with \texttt{sudo -i} then run the installation commands

\begin{verbatim}
sudo -i
echo /opt/etherlab/lib > /etc/ld.so.conf.d/etherlab.conf
/sbin/ldconfig
cp -f /opt/etherlab/lib/pkgconfig/libpdserv.pc /usr/share/pkgconfig/
\end{verbatim}


\subsubsection{Compile a test example}
\label{test-server-for-testmanager}
Go to the \texttt{test} subdirectory.

\begin{verbatim}
cd test
g++ -o test1 test1.cpp \
    -I/opt/etherlab/include -L/opt/etherlab/lib -lpdserv
\end{verbatim}

Run the program
\begin{verbatim}
./test1
\end{verbatim}

\noindent An error message appears but it is not important
\begin{verbatim}
Error loading default config file /etc/opt/etherlab/pdserv.conf :
: File is not readable
\end{verbatim}

Keep this program running because it will be used later to test
TestManager and DLS.
